% Options for packages loaded elsewhere
\PassOptionsToPackage{unicode}{hyperref}
\PassOptionsToPackage{hyphens}{url}
%
\documentclass[
]{article}
\usepackage{amsmath,amssymb}
\usepackage{lmodern}
\usepackage{iftex}
\ifPDFTeX
  \usepackage[T1]{fontenc}
  \usepackage[utf8]{inputenc}
  \usepackage{textcomp} % provide euro and other symbols
\else % if luatex or xetex
  \usepackage{unicode-math}
  \defaultfontfeatures{Scale=MatchLowercase}
  \defaultfontfeatures[\rmfamily]{Ligatures=TeX,Scale=1}
\fi
% Use upquote if available, for straight quotes in verbatim environments
\IfFileExists{upquote.sty}{\usepackage{upquote}}{}
\IfFileExists{microtype.sty}{% use microtype if available
  \usepackage[]{microtype}
  \UseMicrotypeSet[protrusion]{basicmath} % disable protrusion for tt fonts
}{}
\makeatletter
\@ifundefined{KOMAClassName}{% if non-KOMA class
  \IfFileExists{parskip.sty}{%
    \usepackage{parskip}
  }{% else
    \setlength{\parindent}{0pt}
    \setlength{\parskip}{6pt plus 2pt minus 1pt}}
}{% if KOMA class
  \KOMAoptions{parskip=half}}
\makeatother
\usepackage{xcolor}
\IfFileExists{xurl.sty}{\usepackage{xurl}}{} % add URL line breaks if available
\IfFileExists{bookmark.sty}{\usepackage{bookmark}}{\usepackage{hyperref}}
\hypersetup{
  pdftitle={Estudio sobre la Gamificación en UTEC como alternativa a la educación tradicional},
  hidelinks,
  pdfcreator={LaTeX via pandoc}}
\urlstyle{same} % disable monospaced font for URLs
\usepackage[margin=1in]{geometry}
\usepackage{color}
\usepackage{fancyvrb}
\newcommand{\VerbBar}{|}
\newcommand{\VERB}{\Verb[commandchars=\\\{\}]}
\DefineVerbatimEnvironment{Highlighting}{Verbatim}{commandchars=\\\{\}}
% Add ',fontsize=\small' for more characters per line
\usepackage{framed}
\definecolor{shadecolor}{RGB}{248,248,248}
\newenvironment{Shaded}{\begin{snugshade}}{\end{snugshade}}
\newcommand{\AlertTok}[1]{\textcolor[rgb]{0.94,0.16,0.16}{#1}}
\newcommand{\AnnotationTok}[1]{\textcolor[rgb]{0.56,0.35,0.01}{\textbf{\textit{#1}}}}
\newcommand{\AttributeTok}[1]{\textcolor[rgb]{0.77,0.63,0.00}{#1}}
\newcommand{\BaseNTok}[1]{\textcolor[rgb]{0.00,0.00,0.81}{#1}}
\newcommand{\BuiltInTok}[1]{#1}
\newcommand{\CharTok}[1]{\textcolor[rgb]{0.31,0.60,0.02}{#1}}
\newcommand{\CommentTok}[1]{\textcolor[rgb]{0.56,0.35,0.01}{\textit{#1}}}
\newcommand{\CommentVarTok}[1]{\textcolor[rgb]{0.56,0.35,0.01}{\textbf{\textit{#1}}}}
\newcommand{\ConstantTok}[1]{\textcolor[rgb]{0.00,0.00,0.00}{#1}}
\newcommand{\ControlFlowTok}[1]{\textcolor[rgb]{0.13,0.29,0.53}{\textbf{#1}}}
\newcommand{\DataTypeTok}[1]{\textcolor[rgb]{0.13,0.29,0.53}{#1}}
\newcommand{\DecValTok}[1]{\textcolor[rgb]{0.00,0.00,0.81}{#1}}
\newcommand{\DocumentationTok}[1]{\textcolor[rgb]{0.56,0.35,0.01}{\textbf{\textit{#1}}}}
\newcommand{\ErrorTok}[1]{\textcolor[rgb]{0.64,0.00,0.00}{\textbf{#1}}}
\newcommand{\ExtensionTok}[1]{#1}
\newcommand{\FloatTok}[1]{\textcolor[rgb]{0.00,0.00,0.81}{#1}}
\newcommand{\FunctionTok}[1]{\textcolor[rgb]{0.00,0.00,0.00}{#1}}
\newcommand{\ImportTok}[1]{#1}
\newcommand{\InformationTok}[1]{\textcolor[rgb]{0.56,0.35,0.01}{\textbf{\textit{#1}}}}
\newcommand{\KeywordTok}[1]{\textcolor[rgb]{0.13,0.29,0.53}{\textbf{#1}}}
\newcommand{\NormalTok}[1]{#1}
\newcommand{\OperatorTok}[1]{\textcolor[rgb]{0.81,0.36,0.00}{\textbf{#1}}}
\newcommand{\OtherTok}[1]{\textcolor[rgb]{0.56,0.35,0.01}{#1}}
\newcommand{\PreprocessorTok}[1]{\textcolor[rgb]{0.56,0.35,0.01}{\textit{#1}}}
\newcommand{\RegionMarkerTok}[1]{#1}
\newcommand{\SpecialCharTok}[1]{\textcolor[rgb]{0.00,0.00,0.00}{#1}}
\newcommand{\SpecialStringTok}[1]{\textcolor[rgb]{0.31,0.60,0.02}{#1}}
\newcommand{\StringTok}[1]{\textcolor[rgb]{0.31,0.60,0.02}{#1}}
\newcommand{\VariableTok}[1]{\textcolor[rgb]{0.00,0.00,0.00}{#1}}
\newcommand{\VerbatimStringTok}[1]{\textcolor[rgb]{0.31,0.60,0.02}{#1}}
\newcommand{\WarningTok}[1]{\textcolor[rgb]{0.56,0.35,0.01}{\textbf{\textit{#1}}}}
\usepackage{graphicx}
\makeatletter
\def\maxwidth{\ifdim\Gin@nat@width>\linewidth\linewidth\else\Gin@nat@width\fi}
\def\maxheight{\ifdim\Gin@nat@height>\textheight\textheight\else\Gin@nat@height\fi}
\makeatother
% Scale images if necessary, so that they will not overflow the page
% margins by default, and it is still possible to overwrite the defaults
% using explicit options in \includegraphics[width, height, ...]{}
\setkeys{Gin}{width=\maxwidth,height=\maxheight,keepaspectratio}
% Set default figure placement to htbp
\makeatletter
\def\fps@figure{htbp}
\makeatother
\setlength{\emergencystretch}{3em} % prevent overfull lines
\providecommand{\tightlist}{%
  \setlength{\itemsep}{0pt}\setlength{\parskip}{0pt}}
\setcounter{secnumdepth}{-\maxdimen} % remove section numbering
\ifLuaTeX
  \usepackage{selnolig}  % disable illegal ligatures
\fi

\title{Estudio sobre la Gamificación en UTEC como alternativa a la
educación tradicional}
\author{}
\date{\vspace{-2.5em}}

\begin{document}
\maketitle

En este estudio intentamos encontrar si existe una mejora de la calidad
educativa utilizando la \emph{Gamificación} para los estudiantes en
UTEC. Para esto, diseñamos un experimento en el que dos grupos
(seleccionados por los entrevistadores), uno que utiliza la experiencia
tradicional de educación y el otro el cual recibe una educación con un
videojuego llamado ``Pc building simulator''. Ambos experimentos se
diseñaron para durar aproximadamente 15 minutos y se diseñaron para
cubrir el mismo material.

Primero la carga de data y la instalación de los paquetes.

\begin{Shaded}
\begin{Highlighting}[]
\FunctionTok{library}\NormalTok{(readr)}
\FunctionTok{library}\NormalTok{(dplyr)}
\end{Highlighting}
\end{Shaded}

\begin{verbatim}
## 
## Attaching package: 'dplyr'
\end{verbatim}

\begin{verbatim}
## The following objects are masked from 'package:stats':
## 
##     filter, lag
\end{verbatim}

\begin{verbatim}
## The following objects are masked from 'package:base':
## 
##     intersect, setdiff, setequal, union
\end{verbatim}

\begin{Shaded}
\begin{Highlighting}[]
\FunctionTok{library}\NormalTok{(ggplot2)}
\FunctionTok{library}\NormalTok{(lubridate)}
\end{Highlighting}
\end{Shaded}

\begin{verbatim}
## 
## Attaching package: 'lubridate'
\end{verbatim}

\begin{verbatim}
## The following objects are masked from 'package:base':
## 
##     date, intersect, setdiff, union
\end{verbatim}

\begin{Shaded}
\begin{Highlighting}[]
\FunctionTok{library}\NormalTok{(stringr)}

\NormalTok{DF }\OtherTok{\textless{}{-}} \FunctionTok{read\_csv}\NormalTok{(}\StringTok{"data.csv"}\NormalTok{)}
\end{Highlighting}
\end{Shaded}

\begin{verbatim}
## Rows: 85 Columns: 22
\end{verbatim}

\begin{verbatim}
## -- Column specification --------------------------------------------------------
## Delimiter: ","
## chr (17): Submission Date, Ingrese su correo de UTEC!!, Nombre, Apellido, ¿C...
## dbl  (5): ¿Cual es tu edad?, ¿En que ciclo te encuentras?, ¿Cuál es tu nivel...
## 
## i Use `spec()` to retrieve the full column specification for this data.
## i Specify the column types or set `show_col_types = FALSE` to quiet this message.
\end{verbatim}

LIMPIEZA DE TIEMPO!!!

\begin{Shaded}
\begin{Highlighting}[]
\NormalTok{DF}\SpecialCharTok{$}\StringTok{\textasciigrave{}}\AttributeTok{Duración de la experiencia}\StringTok{\textasciigrave{}} \OtherTok{\textless{}{-}} \FunctionTok{gsub}\NormalTok{(}\StringTok{"}\SpecialCharTok{\textbackslash{}\textbackslash{}}\StringTok{."}\NormalTok{, }\StringTok{":"}\NormalTok{, DF}\SpecialCharTok{$}\StringTok{\textasciigrave{}}\AttributeTok{Duración de la experiencia}\StringTok{\textasciigrave{}}\NormalTok{)}
\NormalTok{DF}\SpecialCharTok{$}\StringTok{\textasciigrave{}}\AttributeTok{Duración de la experiencia}\StringTok{\textasciigrave{}} \OtherTok{\textless{}{-}} \FunctionTok{gsub}\NormalTok{(}\StringTok{" "}\NormalTok{, }\StringTok{""}\NormalTok{, DF}\SpecialCharTok{$}\StringTok{\textasciigrave{}}\AttributeTok{Duración de la experiencia}\StringTok{\textasciigrave{}}\NormalTok{)}
\NormalTok{DF}\SpecialCharTok{$}\StringTok{\textasciigrave{}}\AttributeTok{Duración de la experiencia}\StringTok{\textasciigrave{}} \OtherTok{\textless{}{-}} \FunctionTok{gsub}\NormalTok{(}\StringTok{"minutos"}\NormalTok{, }\StringTok{""}\NormalTok{, DF}\SpecialCharTok{$}\StringTok{\textasciigrave{}}\AttributeTok{Duración de la experiencia}\StringTok{\textasciigrave{}}\NormalTok{)}
\NormalTok{DF}\SpecialCharTok{$}\StringTok{\textasciigrave{}}\AttributeTok{Duración de la experiencia}\StringTok{\textasciigrave{}} \OtherTok{\textless{}{-}} \FunctionTok{gsub}\NormalTok{(}\StringTok{"min"}\NormalTok{, }\StringTok{""}\NormalTok{, DF}\SpecialCharTok{$}\StringTok{\textasciigrave{}}\AttributeTok{Duración de la experiencia}\StringTok{\textasciigrave{}}\NormalTok{)}
\NormalTok{DF}\SpecialCharTok{$}\StringTok{\textasciigrave{}}\AttributeTok{Duración de la experiencia}\StringTok{\textasciigrave{}} \OtherTok{\textless{}{-}} \FunctionTok{gsub}\NormalTok{(}\StringTok{"pm"}\NormalTok{, }\StringTok{""}\NormalTok{, DF}\SpecialCharTok{$}\StringTok{\textasciigrave{}}\AttributeTok{Duración de la experiencia}\StringTok{\textasciigrave{}}\NormalTok{)}
\end{Highlighting}
\end{Shaded}

\begin{Shaded}
\begin{Highlighting}[]
\NormalTok{iterator }\OtherTok{=} \DecValTok{1}
\ControlFlowTok{for}\NormalTok{ (palabras }\ControlFlowTok{in}\NormalTok{ DF}\SpecialCharTok{$}\StringTok{\textasciigrave{}}\AttributeTok{Duración de la experiencia}\StringTok{\textasciigrave{}}\NormalTok{)\{}
  \ControlFlowTok{if}\NormalTok{(}\SpecialCharTok{!}\FunctionTok{is.na}\NormalTok{(palabras))\{  }
    \ControlFlowTok{if}\NormalTok{(}\FunctionTok{nchar}\NormalTok{(palabras)}\SpecialCharTok{\textless{}}\DecValTok{3}\NormalTok{)\{}
\NormalTok{            DF}\SpecialCharTok{$}\StringTok{\textasciigrave{}}\AttributeTok{Duración de la experiencia}\StringTok{\textasciigrave{}}\NormalTok{[iterator] }\OtherTok{\textless{}{-}} \FunctionTok{paste0}\NormalTok{(DF}\SpecialCharTok{$}\StringTok{\textasciigrave{}}\AttributeTok{Duración de la experiencia}\StringTok{\textasciigrave{}}\NormalTok{[iterator],}\StringTok{":00"}\NormalTok{)}
            \CommentTok{\#print(iterator)}
\NormalTok{      \}}
\NormalTok{    \}}
          
\NormalTok{    iterator }\OtherTok{=}\NormalTok{ iterator }\SpecialCharTok{+} \DecValTok{1}
\NormalTok{\}}

\NormalTok{DF}\SpecialCharTok{$}\StringTok{\textasciigrave{}}\AttributeTok{Duración de la experiencia}\StringTok{\textasciigrave{}}
\end{Highlighting}
\end{Shaded}

\begin{verbatim}
##  [1] "15:35" "28:25" "16:25" "17:15" "20:00" "16:36" "23:21" "17:20" "16:40"
## [10] "6:40"  "19:25" "15:25" "17:30" "8:00"  "29:49" "15:50" "15:35" "17:35"
## [19] "13:20" "18:30" "15:10" "16:08" "16:20" "18:35" "17:15" "14:42" "25:40"
## [28] "18:10" "24:00" "18:30" "13:47" "18:15" "18:35" "15:08" "16:58" "7:33" 
## [37] "14:25" "17:38" "14:50" "12:05" "15:38" "18:52" "18:52" "25:32" "18:54"
## [46] "20:10" "17:00" "16:00" "20:10" "15:10" "17:00" "15:13" "20:00" "16:41"
## [55] "24:40" "11:10" "14:50" "15:00" "15:00" "17:00" "13:20" "17:20" "19:20"
## [64] "18:15" "17:30" "12:10" "9:30"  "18:30" "27:05" "12:30" "14:33" NA     
## [73] "17:20" "16:00" "15:20" "17:32" "26:20" "16:05" "21:25" "2:00"  NA     
## [82] NA      NA      NA      NA
\end{verbatim}

\begin{Shaded}
\begin{Highlighting}[]
\NormalTok{DF }\OtherTok{\textless{}{-}}\NormalTok{ DF[}\FunctionTok{complete.cases}\NormalTok{(DF}\SpecialCharTok{$}\StringTok{\textasciigrave{}}\AttributeTok{Duración de la experiencia}\StringTok{\textasciigrave{}}\NormalTok{),]}
\NormalTok{Duracion }\OtherTok{\textless{}{-}} \FunctionTok{c}\NormalTok{()}

\ControlFlowTok{for}\NormalTok{ (duracion }\ControlFlowTok{in}\NormalTok{ DF}\SpecialCharTok{$}\StringTok{\textasciigrave{}}\AttributeTok{Duración de la experiencia}\StringTok{\textasciigrave{}}\NormalTok{)}
\NormalTok{\{}
\NormalTok{    minuto }\OtherTok{\textless{}{-}} \FunctionTok{str\_extract}\NormalTok{(duracion, }\StringTok{"[0{-}9]*"}\NormalTok{)}
\NormalTok{    Duracion }\OtherTok{\textless{}{-}} \FunctionTok{append}\NormalTok{(Duracion, }\FunctionTok{strtoi}\NormalTok{(minuto))}
\NormalTok{\}}

\NormalTok{Duracion}
\end{Highlighting}
\end{Shaded}

\begin{verbatim}
##  [1] 15 28 16 17 20 16 23 17 16  6 19 15 17  8 29 15 15 17 13 18 15 16 16 18 17
## [26] 14 25 18 24 18 13 18 18 15 16  7 14 17 14 12 15 18 18 25 18 20 17 16 20 15
## [51] 17 15 20 16 24 11 14 15 15 17 13 17 19 18 17 12  9 18 27 12 14 17 16 15 17
## [76] 26 16 21  2
\end{verbatim}

ESTE CODIGO ARREGLA EL PROBLEMA DE ``LEER DOCUMENTO'' Y ``LEER DOCUMENTO
PDF''

\begin{Shaded}
\begin{Highlighting}[]
\NormalTok{iterator }\OtherTok{=} \DecValTok{1}
\ControlFlowTok{for}\NormalTok{ (palabras }\ControlFlowTok{in}\NormalTok{ DF}\SpecialCharTok{$}\StringTok{\textasciigrave{}}\AttributeTok{Qué experiencia realizaste:}\StringTok{\textasciigrave{}}\NormalTok{)\{}
  \ControlFlowTok{if}\NormalTok{(}\SpecialCharTok{!}\FunctionTok{is.na}\NormalTok{(palabras))\{  }
    \ControlFlowTok{if}\NormalTok{(palabras}\SpecialCharTok{==}\StringTok{"Leer Documento"}\NormalTok{)\{}
\NormalTok{            DF}\SpecialCharTok{$}\StringTok{\textasciigrave{}}\AttributeTok{Qué experiencia realizaste:}\StringTok{\textasciigrave{}}\NormalTok{[iterator] }\OtherTok{\textless{}{-}} \FunctionTok{paste0}\NormalTok{(DF}\SpecialCharTok{$}\StringTok{\textasciigrave{}}\AttributeTok{Qué experiencia realizaste:}\StringTok{\textasciigrave{}}\NormalTok{[iterator],}\StringTok{" PDF"}\NormalTok{)}
            \CommentTok{\#print(iterator)}
\NormalTok{      \}}
\NormalTok{    \}}
          
\NormalTok{    iterator }\OtherTok{=}\NormalTok{ iterator }\SpecialCharTok{+} \DecValTok{1}
\NormalTok{\}}
\NormalTok{DF}\SpecialCharTok{$}\StringTok{\textasciigrave{}}\AttributeTok{Qué experiencia realizaste:}\StringTok{\textasciigrave{}}
\end{Highlighting}
\end{Shaded}

\begin{verbatim}
##  [1] "Leer Documento PDF" "Leer Documento PDF" "Leer Documento PDF"
##  [4] "Leer Documento PDF" "Leer Documento PDF" "Leer Documento PDF"
##  [7] "Jugar PC Simulator" "Jugar PC Simulator" "Jugar PC Simulator"
## [10] "Jugar PC Simulator" "Leer Documento PDF" "Leer Documento PDF"
## [13] "Jugar PC Simulator" "Jugar PC Simulator" "Jugar PC Simulator"
## [16] "Leer Documento PDF" "Jugar PC Simulator" "Jugar PC Simulator"
## [19] "Jugar PC Simulator" "Jugar PC Simulator" "Jugar PC Simulator"
## [22] "Leer Documento PDF" "Jugar PC Simulator" "Leer Documento PDF"
## [25] "Jugar PC Simulator" "Jugar PC Simulator" "Jugar PC Simulator"
## [28] "Jugar PC Simulator" "Jugar PC Simulator" "Jugar PC Simulator"
## [31] "Jugar PC Simulator" "Jugar PC Simulator" "Leer Documento PDF"
## [34] "Jugar PC Simulator" "Jugar PC Simulator" "Jugar PC Simulator"
## [37] "Leer Documento PDF" "Jugar PC Simulator" "Jugar PC Simulator"
## [40] "Leer Documento PDF" "Jugar PC Simulator" "Jugar PC Simulator"
## [43] "Leer Documento PDF" "Jugar PC Simulator" "Jugar PC Simulator"
## [46] "Jugar PC Simulator" "Leer Documento PDF" "Leer Documento PDF"
## [49] "Leer Documento PDF" "Jugar PC Simulator" "Jugar PC Simulator"
## [52] "Jugar PC Simulator" "Leer Documento PDF" "Jugar PC Simulator"
## [55] "Jugar PC Simulator" "Leer Documento PDF" "Leer Documento PDF"
## [58] "Leer Documento PDF" "Leer Documento PDF" "Leer Documento PDF"
## [61] "Jugar PC Simulator" "Jugar PC Simulator" "Leer Documento PDF"
## [64] "Leer Documento PDF" "Leer Documento PDF" "Leer Documento PDF"
## [67] "Leer Documento PDF" "Leer Documento PDF" "Jugar PC Simulator"
## [70] "Leer Documento PDF" "Leer Documento PDF" "Leer Documento PDF"
## [73] "Leer Documento PDF" "Leer Documento PDF" "Jugar PC Simulator"
## [76] "Jugar PC Simulator" "Jugar PC Simulator" "Jugar PC Simulator"
## [79] "Leer Documento PDF"
\end{verbatim}

Para el filtrado hemos decidido requerir de datos solo a ``Qué
experiencia realizaste'', esto es debido a que no es posible realizar el
estudio si no se tiene a cual de los dos grupos pertenece.

Luego hemos calculado las respuestas correctas ``a mano'', hemos
considerado que los NA como respuestas validas pero con valor de 0.

\begin{Shaded}
\begin{Highlighting}[]
\NormalTok{score\_vector }\OtherTok{=} \FunctionTok{c}\NormalTok{()}

\ControlFlowTok{for}\NormalTok{ (ans }\ControlFlowTok{in}\NormalTok{ DF}\SpecialCharTok{$}\StringTok{\textasciigrave{}}\AttributeTok{Regula y proporciona la potencia para que se activen todos los componentes de la PC:}\StringTok{\textasciigrave{}}\NormalTok{)}
\NormalTok{\{}
  \ControlFlowTok{if}\NormalTok{ (}\SpecialCharTok{!}\FunctionTok{is.na}\NormalTok{(ans))}
\NormalTok{  \{}
    \ControlFlowTok{if}\NormalTok{ (ans }\SpecialCharTok{==} \StringTok{"Fuente de alimentación (PSU) o sistema de alimentación"}\NormalTok{)}
\NormalTok{    \{}
\NormalTok{      score\_vector }\OtherTok{\textless{}{-}} \FunctionTok{append}\NormalTok{(score\_vector, }\DecValTok{1}\NormalTok{)}
\NormalTok{    \}}
    \ControlFlowTok{else}
\NormalTok{    \{}
\NormalTok{      score\_vector }\OtherTok{\textless{}{-}} \FunctionTok{append}\NormalTok{(score\_vector, }\DecValTok{0}\NormalTok{)}
\NormalTok{    \}  }
\NormalTok{  \}}
  \ControlFlowTok{else}
\NormalTok{  \{}
\NormalTok{    score\_vector }\OtherTok{\textless{}{-}} \FunctionTok{append}\NormalTok{(score\_vector, }\DecValTok{0}\NormalTok{)}
\NormalTok{  \}}
  
\NormalTok{\}}

\NormalTok{iterator }\OtherTok{=} \DecValTok{1}
\ControlFlowTok{for}\NormalTok{ (ans }\ControlFlowTok{in}\NormalTok{ DF}\SpecialCharTok{$}\StringTok{\textasciigrave{}}\AttributeTok{Cuenta con sus propios procesadores y memoria interna que procesa los datos de la PC y los representa en forma de texto o gráficas complejas como la de los videojuegos. Nos referimos a:}\StringTok{\textasciigrave{}}\NormalTok{)}
\NormalTok{\{}
  \ControlFlowTok{if}\NormalTok{ (}\SpecialCharTok{!}\FunctionTok{is.na}\NormalTok{(ans))}
\NormalTok{  \{}
    \ControlFlowTok{if}\NormalTok{ (ans }\SpecialCharTok{==} \StringTok{"Tarjeta gráfica (GPU)"}\NormalTok{)}
\NormalTok{    \{}
\NormalTok{      score\_vector[iterator] }\OtherTok{=}\NormalTok{ score\_vector[iterator] }\SpecialCharTok{+} \DecValTok{1}
\NormalTok{    \}  }
\NormalTok{  \}}
  
\NormalTok{  iterator }\OtherTok{=}\NormalTok{ iterator }\SpecialCharTok{+} \DecValTok{1}
\NormalTok{\}}


\NormalTok{iterator }\OtherTok{=} \DecValTok{1}
\ControlFlowTok{for}\NormalTok{ (ans }\ControlFlowTok{in}\NormalTok{ DF}\SpecialCharTok{$}\StringTok{\textasciigrave{}}\AttributeTok{Responsable realiza cálculos basados en la interpretación de la información de ciertos componentes y les pasa el resultado a otros. Nos referimos a:}\StringTok{\textasciigrave{}}\NormalTok{)}
\NormalTok{\{}
  \ControlFlowTok{if}\NormalTok{ (}\SpecialCharTok{!}\FunctionTok{is.na}\NormalTok{(ans))}
\NormalTok{  \{}
    \ControlFlowTok{if}\NormalTok{ (ans }\SpecialCharTok{==} \StringTok{"Unidad central de procesamiento o Procesador (CPU)"}\NormalTok{)}
\NormalTok{    \{}
\NormalTok{      score\_vector[iterator] }\OtherTok{\textless{}{-}}\NormalTok{ score\_vector[iterator] }\SpecialCharTok{+} \DecValTok{1}
\NormalTok{    \}  }
\NormalTok{  \}}
  
\NormalTok{  iterator }\OtherTok{=}\NormalTok{ iterator }\SpecialCharTok{+} \DecValTok{1}
\NormalTok{\}}


\NormalTok{iterator }\OtherTok{=} \DecValTok{1}
\ControlFlowTok{for}\NormalTok{ (ans }\ControlFlowTok{in}\NormalTok{ DF}\SpecialCharTok{$}\StringTok{\textasciigrave{}}\AttributeTok{Almacena todos los datos del sistema operativo, los programas y las fotos, la música y los vídeos del usuario. Es decir, es el dispositivo de almacenamiento del ordenador.}\StringTok{\textasciigrave{}}\NormalTok{)}
\NormalTok{\{}
  \ControlFlowTok{if}\NormalTok{ (}\SpecialCharTok{!}\FunctionTok{is.na}\NormalTok{(ans))}
\NormalTok{  \{}
    \ControlFlowTok{if}\NormalTok{ (ans }\SpecialCharTok{==} \StringTok{"Disco Duro"}\NormalTok{)}
\NormalTok{    \{}
\NormalTok{      score\_vector[iterator] }\OtherTok{\textless{}{-}}\NormalTok{ score\_vector[iterator] }\SpecialCharTok{+} \DecValTok{1}
\NormalTok{    \}  }
\NormalTok{  \}}
  
\NormalTok{  iterator }\OtherTok{=}\NormalTok{ iterator }\SpecialCharTok{+} \DecValTok{1}
\NormalTok{\}}

\NormalTok{iterator }\OtherTok{=} \DecValTok{1}
\ControlFlowTok{for}\NormalTok{ (ans }\ControlFlowTok{in}\NormalTok{ DF}\SpecialCharTok{$}\StringTok{\textasciigrave{}}\AttributeTok{¿Dónde se conecta o instala la memoria (RAM)?}\StringTok{\textasciigrave{}}\NormalTok{)}
\NormalTok{\{}
  \ControlFlowTok{if}\NormalTok{ (}\SpecialCharTok{!}\FunctionTok{is.na}\NormalTok{(ans))}
\NormalTok{  \{}
    \ControlFlowTok{if}\NormalTok{ (ans }\SpecialCharTok{==} \StringTok{"Placa base (Motherboard)"}\NormalTok{)}
\NormalTok{    \{}
\NormalTok{      score\_vector[iterator] }\OtherTok{\textless{}{-}}\NormalTok{ score\_vector[iterator] }\SpecialCharTok{+} \DecValTok{1}
\NormalTok{    \}  }
\NormalTok{  \}}
  
\NormalTok{  iterator }\OtherTok{=}\NormalTok{ iterator }\SpecialCharTok{+} \DecValTok{1}
\NormalTok{\}}

\NormalTok{iterator }\OtherTok{=} \DecValTok{1}
\ControlFlowTok{for}\NormalTok{ (ans }\ControlFlowTok{in}\NormalTok{ DF}\SpecialCharTok{$}\StringTok{\textasciigrave{}}\AttributeTok{¿Dónde se coloca la pasta térmica?}\StringTok{\textasciigrave{}}\NormalTok{)}
\NormalTok{\{}
  \ControlFlowTok{if}\NormalTok{ (}\SpecialCharTok{!}\FunctionTok{is.na}\NormalTok{(ans))}
\NormalTok{  \{}
    \ControlFlowTok{if}\NormalTok{ (ans }\SpecialCharTok{==} \StringTok{"Unidad central de procesamiento o Procesador (CPU)"}\NormalTok{)}
\NormalTok{    \{}
\NormalTok{      score\_vector[iterator] }\OtherTok{\textless{}{-}}\NormalTok{ score\_vector[iterator] }\SpecialCharTok{+} \DecValTok{1}
\NormalTok{    \}  }
\NormalTok{  \}}
  
\NormalTok{  iterator }\OtherTok{=}\NormalTok{ iterator }\SpecialCharTok{+} \DecValTok{1}
\NormalTok{\}}

\NormalTok{iterator }\OtherTok{=} \DecValTok{1}
\ControlFlowTok{for}\NormalTok{ (ans }\ControlFlowTok{in}\NormalTok{ DF}\SpecialCharTok{$}\StringTok{\textasciigrave{}}\AttributeTok{¿A que parte se conecta el panel frontal de la caja (botón de encendido/apagado, USB, conector de audio, etc)?}\StringTok{\textasciigrave{}}\NormalTok{)}
\NormalTok{\{}
  \ControlFlowTok{if}\NormalTok{ (}\SpecialCharTok{!}\FunctionTok{is.na}\NormalTok{(ans))}
\NormalTok{  \{}
    \ControlFlowTok{if}\NormalTok{ (ans }\SpecialCharTok{==} \StringTok{"Placa base (Motherboard)"}\NormalTok{)}
\NormalTok{    \{}
\NormalTok{      score\_vector[iterator] }\OtherTok{\textless{}{-}}\NormalTok{ score\_vector[iterator] }\SpecialCharTok{+} \DecValTok{1}
\NormalTok{    \}  }
\NormalTok{  \}}
  
\NormalTok{  iterator }\OtherTok{=}\NormalTok{ iterator }\SpecialCharTok{+} \DecValTok{1}
\NormalTok{\}}



\NormalTok{DF\_scored }\OtherTok{\textless{}{-}} \FunctionTok{cbind}\NormalTok{(DF, score\_vector)}
\end{Highlighting}
\end{Shaded}

Luego de medir el puntaje hemos dividido las respuestas en los dos
grupos correspondientes. Por ultimo hemos hecho una resta de filas
debido a que hubo un imbalance de experimentos. Notar que aunque se
quitaron filas y se puede considerar \emph{MANIPULACIÓN DE DATOS} dichas
columnas fueron removidas sin ningún tipo de ``filtro'' en particular
mas que el orden en el que llegaron. Por lo que, se puede considerar
``Aleatorio'' las filas que se quitaron.

\begin{Shaded}
\begin{Highlighting}[]
\NormalTok{class\_learning }\OtherTok{\textless{}{-}}\NormalTok{ DF\_scored[DF\_scored}\SpecialCharTok{$}\StringTok{\textasciigrave{}}\AttributeTok{Qué experiencia realizaste:}\StringTok{\textasciigrave{}} \SpecialCharTok{==} \StringTok{\textquotesingle{}Leer Documento PDF\textquotesingle{}}\NormalTok{,]}



\NormalTok{game\_learning }\OtherTok{\textless{}{-}}\NormalTok{ DF\_scored[DF\_scored}\SpecialCharTok{$}\StringTok{\textasciigrave{}}\AttributeTok{Qué experiencia realizaste:}\StringTok{\textasciigrave{}} \SpecialCharTok{==} \StringTok{\textquotesingle{}Jugar PC Simulator\textquotesingle{}}\NormalTok{,]}

\NormalTok{game\_learning\_cortado }\OtherTok{\textless{}{-}} \FunctionTok{head}\NormalTok{(game\_learning, }\FunctionTok{nrow}\NormalTok{(class\_learning) }\SpecialCharTok{{-}} \FunctionTok{nrow}\NormalTok{(game\_learning))}


\NormalTok{data }\OtherTok{\textless{}{-}} \FunctionTok{data.frame}\NormalTok{(class\_learning}\SpecialCharTok{$}\NormalTok{score\_vector, game\_learning\_cortado}\SpecialCharTok{$}\NormalTok{score\_vector)}
\FunctionTok{boxplot}\NormalTok{(data,}\AttributeTok{names =} \FunctionTok{c}\NormalTok{(}\StringTok{"PDF"}\NormalTok{, }\StringTok{"Juego"}\NormalTok{))}
\end{Highlighting}
\end{Shaded}

\includegraphics{proceso_files/figure-latex/unnamed-chunk-6-1.pdf}

Tambien podemos medir la nota contra la experiencia de cada alumno

Grafica para los que hirieron clase de su nivel de experiencia armando
pc contra su puntaje

\begin{Shaded}
\begin{Highlighting}[]
\NormalTok{score\_mean }\OtherTok{\textless{}{-}} \FunctionTok{c}\NormalTok{(}\DecValTok{0}\NormalTok{,}\DecValTok{0}\NormalTok{,}\DecValTok{0}\NormalTok{,}\DecValTok{0}\NormalTok{,}\DecValTok{0}\NormalTok{)}
\NormalTok{number\_to\_divide }\OtherTok{\textless{}{-}} \FunctionTok{c}\NormalTok{(}\DecValTok{0}\NormalTok{,}\DecValTok{0}\NormalTok{,}\DecValTok{0}\NormalTok{,}\DecValTok{0}\NormalTok{,}\DecValTok{0}\NormalTok{)}

\ControlFlowTok{for}\NormalTok{ (row }\ControlFlowTok{in} \DecValTok{1}\SpecialCharTok{:}\FunctionTok{nrow}\NormalTok{(class\_learning))}
\NormalTok{\{}
\NormalTok{  score\_mean[class\_learning[row, }\StringTok{"¿Cuál es tu nivel de experiencia armando computadoras?"}\NormalTok{]] }\OtherTok{\textless{}{-}}\NormalTok{ score\_mean[class\_learning[row, }\StringTok{"¿Cuál es tu nivel de experiencia armando computadoras?"}\NormalTok{]] }\SpecialCharTok{+}\NormalTok{ class\_learning[row, }\StringTok{"score\_vector"}\NormalTok{]}
\NormalTok{  number\_to\_divide[class\_learning[row, }\StringTok{"¿Cuál es tu nivel de experiencia armando computadoras?"}\NormalTok{]] }\OtherTok{\textless{}{-}}\NormalTok{ number\_to\_divide[class\_learning[row, }\StringTok{"¿Cuál es tu nivel de experiencia armando computadoras?"}\NormalTok{]] }\SpecialCharTok{+} \DecValTok{1}
\NormalTok{\}}

\ControlFlowTok{for}\NormalTok{ (iterator }\ControlFlowTok{in} \DecValTok{1}\SpecialCharTok{:}\DecValTok{5}\NormalTok{)}
\NormalTok{\{}
\NormalTok{  score\_mean[iterator] }\OtherTok{=}\NormalTok{ score\_mean[iterator]}\SpecialCharTok{/}\NormalTok{number\_to\_divide[iterator]}
\NormalTok{\}}


\NormalTok{possible }\OtherTok{\textless{}{-}} \FunctionTok{c}\NormalTok{(}\DecValTok{1}\NormalTok{,}\DecValTok{2}\NormalTok{,}\DecValTok{3}\NormalTok{,}\DecValTok{4}\NormalTok{,}\DecValTok{5}\NormalTok{)}

\FunctionTok{plot}\NormalTok{(possible, score\_mean, }\AttributeTok{xlab =} \StringTok{"Nivel De Experiencia"}\NormalTok{, }\AttributeTok{ylab =} \StringTok{"Promedio de puntaje"}\NormalTok{)}
\end{Highlighting}
\end{Shaded}

\includegraphics{proceso_files/figure-latex/unnamed-chunk-7-1.pdf}

Grafica para los que hirieron videojuego de su nivel de experiencia
armando pc contra su puntaje

\begin{Shaded}
\begin{Highlighting}[]
\NormalTok{score\_mean }\OtherTok{\textless{}{-}}\NormalTok{ score\_mean[}\FunctionTok{complete.cases}\NormalTok{(score\_mean)]}
\NormalTok{score\_mean }\OtherTok{\textless{}{-}} \FunctionTok{c}\NormalTok{(}\DecValTok{0}\NormalTok{,}\DecValTok{0}\NormalTok{,}\DecValTok{0}\NormalTok{,}\DecValTok{0}\NormalTok{,}\DecValTok{0}\NormalTok{)}
\NormalTok{number\_to\_divide }\OtherTok{\textless{}{-}} \FunctionTok{c}\NormalTok{(}\DecValTok{0}\NormalTok{,}\DecValTok{0}\NormalTok{,}\DecValTok{0}\NormalTok{,}\DecValTok{0}\NormalTok{,}\DecValTok{0}\NormalTok{)}

\ControlFlowTok{for}\NormalTok{ (row }\ControlFlowTok{in} \DecValTok{1}\SpecialCharTok{:}\FunctionTok{nrow}\NormalTok{(game\_learning\_cortado))}
\NormalTok{\{}
\NormalTok{  score\_mean[game\_learning\_cortado[row, }\StringTok{"¿Cuál es tu nivel de experiencia armando computadoras?"}\NormalTok{]] }\OtherTok{\textless{}{-}}\NormalTok{ score\_mean[game\_learning\_cortado[row, }\StringTok{"¿Cuál es tu nivel de experiencia armando computadoras?"}\NormalTok{]] }\SpecialCharTok{+}\NormalTok{ game\_learning\_cortado[row, }\StringTok{"score\_vector"}\NormalTok{]}
\NormalTok{  number\_to\_divide[game\_learning\_cortado[row, }\StringTok{"¿Cuál es tu nivel de experiencia armando computadoras?"}\NormalTok{]] }\OtherTok{\textless{}{-}}\NormalTok{ number\_to\_divide[game\_learning\_cortado[row, }\StringTok{"¿Cuál es tu nivel de experiencia armando computadoras?"}\NormalTok{]] }\SpecialCharTok{+} \DecValTok{1}
\NormalTok{\}}

\ControlFlowTok{for}\NormalTok{ (iterator }\ControlFlowTok{in} \DecValTok{1}\SpecialCharTok{:}\DecValTok{5}\NormalTok{)}
\NormalTok{\{}
\NormalTok{  score\_mean[iterator] }\OtherTok{=}\NormalTok{ score\_mean[iterator]}\SpecialCharTok{/}\NormalTok{number\_to\_divide[iterator]}
\NormalTok{\}}

\NormalTok{possible }\OtherTok{\textless{}{-}} \FunctionTok{c}\NormalTok{(}\DecValTok{1}\NormalTok{,}\DecValTok{2}\NormalTok{,}\DecValTok{3}\NormalTok{,}\DecValTok{4}\NormalTok{,}\DecValTok{5}\NormalTok{)}

\FunctionTok{plot}\NormalTok{(possible, score\_mean, }\AttributeTok{xlab =} \StringTok{"Nivel De Experiencia"}\NormalTok{, }\AttributeTok{ylab =} \StringTok{"Promedio de puntaje"}\NormalTok{)}
\end{Highlighting}
\end{Shaded}

\includegraphics{proceso_files/figure-latex/unnamed-chunk-8-1.pdf}

Grafica de los que jugaron clase de su nivel de experiencia con los
componentes de una computadora vs su puntaje

\begin{Shaded}
\begin{Highlighting}[]
\NormalTok{score\_mean }\OtherTok{\textless{}{-}}\NormalTok{ score\_mean[}\FunctionTok{complete.cases}\NormalTok{(score\_mean)]}
\NormalTok{score\_mean }\OtherTok{\textless{}{-}} \FunctionTok{c}\NormalTok{(}\DecValTok{0}\NormalTok{,}\DecValTok{0}\NormalTok{,}\DecValTok{0}\NormalTok{,}\DecValTok{0}\NormalTok{,}\DecValTok{0}\NormalTok{)}
\NormalTok{number\_to\_divide }\OtherTok{\textless{}{-}} \FunctionTok{c}\NormalTok{(}\DecValTok{0}\NormalTok{,}\DecValTok{0}\NormalTok{,}\DecValTok{0}\NormalTok{,}\DecValTok{0}\NormalTok{,}\DecValTok{0}\NormalTok{)}

\ControlFlowTok{for}\NormalTok{ (row }\ControlFlowTok{in} \DecValTok{1}\SpecialCharTok{:}\FunctionTok{nrow}\NormalTok{(class\_learning))}
\NormalTok{\{}
\NormalTok{  score\_mean[class\_learning[row, }\StringTok{"¿Cuál es tu nivel de conocimiento (previo a la experiencia) sobre componentes de computadora?"}\NormalTok{]] }\OtherTok{\textless{}{-}}\NormalTok{ score\_mean[class\_learning[row, }\StringTok{"¿Cuál es tu nivel de conocimiento (previo a la experiencia) sobre componentes de computadora?"}\NormalTok{]] }\SpecialCharTok{+}\NormalTok{ class\_learning[row, }\StringTok{"score\_vector"}\NormalTok{]}
\NormalTok{  number\_to\_divide[class\_learning[row, }\StringTok{"¿Cuál es tu nivel de conocimiento (previo a la experiencia) sobre componentes de computadora?"}\NormalTok{]] }\OtherTok{\textless{}{-}}\NormalTok{ number\_to\_divide[class\_learning[row, }\StringTok{"¿Cuál es tu nivel de conocimiento (previo a la experiencia) sobre componentes de computadora?"}\NormalTok{]] }\SpecialCharTok{+} \DecValTok{1}
\NormalTok{\}}

\ControlFlowTok{for}\NormalTok{ (iterator }\ControlFlowTok{in} \DecValTok{1}\SpecialCharTok{:}\DecValTok{5}\NormalTok{)}
\NormalTok{\{}
\NormalTok{  score\_mean[iterator] }\OtherTok{=}\NormalTok{ score\_mean[iterator]}\SpecialCharTok{/}\NormalTok{number\_to\_divide[iterator]}
\NormalTok{\}}

\NormalTok{possible }\OtherTok{\textless{}{-}} \FunctionTok{c}\NormalTok{(}\DecValTok{1}\NormalTok{,}\DecValTok{2}\NormalTok{,}\DecValTok{3}\NormalTok{,}\DecValTok{4}\NormalTok{,}\DecValTok{5}\NormalTok{)}

\FunctionTok{plot}\NormalTok{(possible,score\_mean, }\AttributeTok{xlab =} \StringTok{"Nivel De Experiencia (Componentes)"}\NormalTok{, }\AttributeTok{ylab =} \StringTok{"Promedio de puntaje"}\NormalTok{, }\AttributeTok{yaxp=}\FunctionTok{c}\NormalTok{(}\DecValTok{0}\NormalTok{,}\DecValTok{7}\NormalTok{,}\DecValTok{20}\NormalTok{))}
\end{Highlighting}
\end{Shaded}

\includegraphics{proceso_files/figure-latex/unnamed-chunk-9-1.pdf}
Grafica de los que jugaron videojuego y su nivel de experiencia con los
componentes de una computadora vs su puntaje

\begin{Shaded}
\begin{Highlighting}[]
\NormalTok{score\_mean }\OtherTok{\textless{}{-}}\NormalTok{ score\_mean[}\FunctionTok{complete.cases}\NormalTok{(score\_mean)]}
\NormalTok{score\_mean }\OtherTok{\textless{}{-}} \FunctionTok{c}\NormalTok{(}\DecValTok{0}\NormalTok{,}\DecValTok{0}\NormalTok{,}\DecValTok{0}\NormalTok{,}\DecValTok{0}\NormalTok{,}\DecValTok{0}\NormalTok{)}
\NormalTok{number\_to\_divide }\OtherTok{\textless{}{-}} \FunctionTok{c}\NormalTok{(}\DecValTok{0}\NormalTok{,}\DecValTok{0}\NormalTok{,}\DecValTok{0}\NormalTok{,}\DecValTok{0}\NormalTok{,}\DecValTok{0}\NormalTok{)}

\ControlFlowTok{for}\NormalTok{ (row }\ControlFlowTok{in} \DecValTok{1}\SpecialCharTok{:}\FunctionTok{nrow}\NormalTok{(game\_learning\_cortado))}
\NormalTok{\{}
\NormalTok{  score\_mean[game\_learning\_cortado[row, }\StringTok{"¿Cuál es tu nivel de conocimiento (previo a la experiencia) sobre componentes de computadora?"}\NormalTok{]] }\OtherTok{\textless{}{-}}\NormalTok{ score\_mean[game\_learning\_cortado[row, }\StringTok{"¿Cuál es tu nivel de conocimiento (previo a la experiencia) sobre componentes de computadora?"}\NormalTok{]] }\SpecialCharTok{+}\NormalTok{ game\_learning\_cortado[row, }\StringTok{"score\_vector"}\NormalTok{]}
\NormalTok{  number\_to\_divide[game\_learning\_cortado[row, }\StringTok{"¿Cuál es tu nivel de conocimiento (previo a la experiencia) sobre componentes de computadora?"}\NormalTok{]] }\OtherTok{\textless{}{-}}\NormalTok{ number\_to\_divide[game\_learning\_cortado[row, }\StringTok{"¿Cuál es tu nivel de conocimiento (previo a la experiencia) sobre componentes de computadora?"}\NormalTok{]] }\SpecialCharTok{+} \DecValTok{1}
\NormalTok{\}}

\ControlFlowTok{for}\NormalTok{ (iterator }\ControlFlowTok{in} \DecValTok{1}\SpecialCharTok{:}\DecValTok{5}\NormalTok{)}
\NormalTok{\{}
\NormalTok{  score\_mean[iterator] }\OtherTok{=}\NormalTok{ score\_mean[iterator]}\SpecialCharTok{/}\NormalTok{number\_to\_divide[iterator]}
\NormalTok{\}}

\NormalTok{possible }\OtherTok{\textless{}{-}} \FunctionTok{c}\NormalTok{(}\DecValTok{1}\NormalTok{,}\DecValTok{2}\NormalTok{,}\DecValTok{3}\NormalTok{,}\DecValTok{4}\NormalTok{,}\DecValTok{5}\NormalTok{)}

\FunctionTok{plot}\NormalTok{(possible,score\_mean, }\AttributeTok{xlab =} \StringTok{"Nivel De Experiencia (Componentes)"}\NormalTok{, }\AttributeTok{ylab =} \StringTok{"Promedio de puntaje"}\NormalTok{, }\AttributeTok{yaxp=}\FunctionTok{c}\NormalTok{(}\DecValTok{0}\NormalTok{,}\DecValTok{7}\NormalTok{,}\DecValTok{20}\NormalTok{))}
\end{Highlighting}
\end{Shaded}

\includegraphics{proceso_files/figure-latex/unnamed-chunk-10-1.pdf}

Tambien podemos ver el tiempo vs nota

\begin{Shaded}
\begin{Highlighting}[]
\NormalTok{seconds }\OtherTok{\textless{}{-}} \FunctionTok{as.numeric}\NormalTok{(}\FunctionTok{as.period}\NormalTok{(}\FunctionTok{ms}\NormalTok{(DF\_scored}\SpecialCharTok{$}\StringTok{\textasciigrave{}}\AttributeTok{Duración de la experiencia}\StringTok{\textasciigrave{}}\NormalTok{), }\AttributeTok{unit =} \StringTok{"sec"}\NormalTok{))}
\NormalTok{DF\_scored }\OtherTok{\textless{}{-}} \FunctionTok{cbind}\NormalTok{(DF\_scored, seconds)}
\end{Highlighting}
\end{Shaded}

\begin{Shaded}
\begin{Highlighting}[]
\FunctionTok{plot}\NormalTok{(DF\_scored}\SpecialCharTok{$}\NormalTok{seconds, DF\_scored}\SpecialCharTok{$}\NormalTok{score\_vector, }\AttributeTok{pch =} \DecValTok{20}\NormalTok{, }\AttributeTok{col =} \FunctionTok{rgb}\NormalTok{(}\DecValTok{0}\NormalTok{, }\DecValTok{0}\NormalTok{, }\DecValTok{0}\NormalTok{, }\FloatTok{0.2}\NormalTok{), }\AttributeTok{xlab=}\StringTok{"Tiempo de aprendizaje (Segundos)"}\NormalTok{, }\AttributeTok{ylab=}\StringTok{"Puntaje"}\NormalTok{)}
\end{Highlighting}
\end{Shaded}

\includegraphics{proceso_files/figure-latex/unnamed-chunk-12-1.pdf}

No parece haber una influencia del tiempo

Con 31 participantes por grupo (62 total) no hemos notado una diferencia
significante entre el grupo que paso la experiencia de gamificación y el
que no. Como idea intentemos ahora tomar todos los NA como datos no
validos

\begin{Shaded}
\begin{Highlighting}[]
\FunctionTok{unique}\NormalTok{(DF}\SpecialCharTok{$}\StringTok{\textasciigrave{}}\AttributeTok{Qué experiencia realizaste:}\StringTok{\textasciigrave{}}\NormalTok{)}
\end{Highlighting}
\end{Shaded}

\begin{verbatim}
## [1] "Leer Documento PDF" "Jugar PC Simulator"
\end{verbatim}

INICIO GANT

\begin{Shaded}
\begin{Highlighting}[]
\FunctionTok{library}\NormalTok{(tidyverse)}
\end{Highlighting}
\end{Shaded}

\begin{verbatim}
## -- Attaching packages --------------------------------------- tidyverse 1.3.1 --
\end{verbatim}

\begin{verbatim}
## v tibble  3.1.6     v purrr   0.3.4
## v tidyr   1.2.0     v forcats 0.5.1
\end{verbatim}

\begin{verbatim}
## -- Conflicts ------------------------------------------ tidyverse_conflicts() --
## x lubridate::as.difftime() masks base::as.difftime()
## x lubridate::date()        masks base::date()
## x dplyr::filter()          masks stats::filter()
## x lubridate::intersect()   masks base::intersect()
## x dplyr::lag()             masks stats::lag()
## x lubridate::setdiff()     masks base::setdiff()
## x lubridate::union()       masks base::union()
\end{verbatim}

\begin{Shaded}
\begin{Highlighting}[]
\FunctionTok{library}\NormalTok{(lubridate)}
\NormalTok{gantt }\OtherTok{\textless{}{-}} \FunctionTok{read.csv}\NormalTok{(}\StringTok{"gant.csv"}\NormalTok{, }\AttributeTok{h=}\NormalTok{T)}
\end{Highlighting}
\end{Shaded}

\begin{Shaded}
\begin{Highlighting}[]
\NormalTok{gantt}\SpecialCharTok{$}\NormalTok{Inicio }\OtherTok{\textless{}{-}} \FunctionTok{gsub}\NormalTok{(}\StringTok{"}\SpecialCharTok{\textbackslash{}\textbackslash{}}\StringTok{."}\NormalTok{, }\StringTok{"{-}"}\NormalTok{, gantt}\SpecialCharTok{$}\NormalTok{Inicio)}
\NormalTok{gantt}\SpecialCharTok{$}\NormalTok{Fin }\OtherTok{\textless{}{-}} \FunctionTok{gsub}\NormalTok{(}\StringTok{"}\SpecialCharTok{\textbackslash{}\textbackslash{}}\StringTok{."}\NormalTok{, }\StringTok{"{-}"}\NormalTok{, gantt}\SpecialCharTok{$}\NormalTok{Fin)}
\NormalTok{gantt}\SpecialCharTok{$}\NormalTok{Inicio }\OtherTok{\textless{}{-}} \FunctionTok{ymd}\NormalTok{(gantt}\SpecialCharTok{$}\NormalTok{Inicio)}
\NormalTok{gantt}\SpecialCharTok{$}\NormalTok{Fin }\OtherTok{\textless{}{-}} \FunctionTok{ymd}\NormalTok{(gantt}\SpecialCharTok{$}\NormalTok{Fin)}
\NormalTok{gantt.melt }\OtherTok{\textless{}{-}}\NormalTok{ gantt}\SpecialCharTok{\%\textgreater{}\%}
\NormalTok{  tidyr}\SpecialCharTok{::}\FunctionTok{pivot\_longer}\NormalTok{(}\AttributeTok{col =} \FunctionTok{c}\NormalTok{(Inicio,Fin))}
\NormalTok{gantt.melt}
\end{Highlighting}
\end{Shaded}

\begin{verbatim}
## # A tibble: 28 x 7
##     Item Actividad                 Elemento.del.Pr~ Color X     name  value     
##    <int> <chr>                     <chr>            <chr> <lgl> <chr> <date>    
##  1     1 Primer contacto con equi~ Entregable P0    red   NA    Inic~ 2022-03-21
##  2     1 Primer contacto con equi~ Entregable P0    red   NA    Fin   2022-03-29
##  3     2 Reunion para definir tem~ Entregable P0    red   NA    Inic~ 2022-04-03
##  4     2 Reunion para definir tem~ Entregable P0    red   NA    Fin   2022-04-06
##  5     3 Redaccion de informe par~ Entregable P0    red   NA    Inic~ 2022-04-06
##  6     3 Redaccion de informe par~ Entregable P0    red   NA    Fin   2022-04-08
##  7     4 Espera de Feeback de ent~ Entregable P0    red   NA    Inic~ 2022-04-09
##  8     4 Espera de Feeback de ent~ Entregable P0    red   NA    Fin   2022-04-21
##  9     5 Reunion para coordinar e~ Tema de Investi~ blue  NA    Inic~ 2022-04-28
## 10     5 Reunion para coordinar e~ Tema de Investi~ blue  NA    Fin   2022-04-30
## # ... with 18 more rows
\end{verbatim}

\begin{Shaded}
\begin{Highlighting}[]
\NormalTok{today }\OtherTok{\textless{}{-}} \FunctionTok{as.Date}\NormalTok{(}\StringTok{\textquotesingle{}2022{-}05{-}06\textquotesingle{}}\NormalTok{)}
\CommentTok{\#Sys.Date()}

\NormalTok{p1 }\OtherTok{\textless{}{-}} \FunctionTok{ggplot}\NormalTok{(gantt.melt,}\FunctionTok{aes}\NormalTok{(}\AttributeTok{x =}\NormalTok{ value,}\AttributeTok{y =}\NormalTok{ Actividad,}\AttributeTok{colour =} \StringTok{"red"}\NormalTok{))}
\NormalTok{p1 }\OtherTok{\textless{}{-}}\NormalTok{ p1 }\SpecialCharTok{+} \FunctionTok{geom\_line}\NormalTok{(}\AttributeTok{alpha=} \FloatTok{0.5}\NormalTok{, }\AttributeTok{size =} \DecValTok{7}\NormalTok{)}
\NormalTok{p1 }\OtherTok{\textless{}{-}}\NormalTok{ p1 }\SpecialCharTok{+} \FunctionTok{geom\_label}\NormalTok{(}\FunctionTok{aes}\NormalTok{(}\AttributeTok{label=}\FunctionTok{format}\NormalTok{(value,}\StringTok{"\%d \%b"}\NormalTok{)),}\AttributeTok{vjust =} \SpecialCharTok{{-}}\FloatTok{0.5}\NormalTok{, }\AttributeTok{angle =} \DecValTok{45}\NormalTok{, }\AttributeTok{size =} \DecValTok{3}\NormalTok{, }\AttributeTok{color =} \StringTok{"black"}\NormalTok{)}
\NormalTok{p1 }\OtherTok{\textless{}{-}}\NormalTok{ p1 }\SpecialCharTok{+} \FunctionTok{theme\_bw}\NormalTok{()}
\NormalTok{p1 }\OtherTok{\textless{}{-}}\NormalTok{ p1 }\SpecialCharTok{+} \FunctionTok{geom\_vline}\NormalTok{(}\AttributeTok{xintercept =}\NormalTok{ today,}\AttributeTok{color =} \StringTok{"grey"}\NormalTok{, }\AttributeTok{size =} \DecValTok{2}\NormalTok{, }\AttributeTok{alpha =} \FloatTok{0.5}\NormalTok{ )}
\NormalTok{p1 }\OtherTok{\textless{}{-}}\NormalTok{ p1 }\SpecialCharTok{+} \FunctionTok{labs}\NormalTok{(}\AttributeTok{title =} \StringTok{"Gant chart"}\NormalTok{)}
\NormalTok{p1 }\OtherTok{\textless{}{-}}\NormalTok{ p1 }\SpecialCharTok{+} \FunctionTok{labs}\NormalTok{(}\AttributeTok{subtitle =} \StringTok{"XDD"}\NormalTok{)}
\NormalTok{p1 }\OtherTok{\textless{}{-}}\NormalTok{ p1 }\SpecialCharTok{+} \FunctionTok{labs}\NormalTok{(}\AttributeTok{caption =} \StringTok{"\#XD"}\NormalTok{)}
\NormalTok{p1 }\OtherTok{\textless{}{-}}\NormalTok{ p1 }\SpecialCharTok{+} \FunctionTok{labs}\NormalTok{(}\AttributeTok{x=}\StringTok{"Date"}\NormalTok{)}
\NormalTok{p1 }\OtherTok{\textless{}{-}}\NormalTok{ p1 }\SpecialCharTok{+}  \FunctionTok{labs}\NormalTok{ (}\AttributeTok{y =} \StringTok{"Items"}\NormalTok{)}
\NormalTok{p1 }\OtherTok{\textless{}{-}}\NormalTok{ p1 }\SpecialCharTok{+}  \FunctionTok{scale\_color\_manual}\NormalTok{(}\AttributeTok{values =} \FunctionTok{c}\NormalTok{(}\StringTok{"red"}\NormalTok{,}\StringTok{"blue"}\NormalTok{))}
\NormalTok{p1 }\OtherTok{\textless{}{-}}\NormalTok{ p1 }\SpecialCharTok{+}  \FunctionTok{theme}\NormalTok{(}\AttributeTok{legend.position =} \StringTok{"none"}\NormalTok{)}
\NormalTok{p1 }\OtherTok{\textless{}{-}}\NormalTok{ p1 }\SpecialCharTok{+} \FunctionTok{scale\_x\_date}\NormalTok{(}\AttributeTok{name =} \StringTok{"Dates"}
\NormalTok{                        , }\AttributeTok{date\_labels =} \StringTok{"\%d \%b"}
\NormalTok{                        , }\AttributeTok{date\_breaks =} \StringTok{"1 week"}
\NormalTok{                        , }\AttributeTok{minor\_breaks =} \StringTok{"2 day"}
\NormalTok{                        , }\AttributeTok{sec.axis =} \FunctionTok{dup\_axis}\NormalTok{(}\AttributeTok{name =} \StringTok{"week number"}\NormalTok{,}\AttributeTok{labels =}\NormalTok{   scales}\SpecialCharTok{::}\FunctionTok{date\_format}\NormalTok{(}\StringTok{"\%W"}\NormalTok{))}
\NormalTok{                        )}
  
\NormalTok{p1}
\end{Highlighting}
\end{Shaded}

\includegraphics{proceso_files/figure-latex/unnamed-chunk-16-1.pdf}

\begin{Shaded}
\begin{Highlighting}[]
\NormalTok{gantt }\OtherTok{\textless{}{-}} \FunctionTok{read.csv}\NormalTok{(}\StringTok{"gant.csv"}\NormalTok{, }\AttributeTok{h=}\NormalTok{T)}
\NormalTok{gantt}
\end{Highlighting}
\end{Shaded}

\begin{verbatim}
##    Item                                                        Actividad
## 1     1                            Primer contacto con equipo de trabajo
## 2     2        Reunion para definir temas de posible estudio estadistico
## 3     3                       Redaccion de informe para el entregable P0
## 4     4                               Espera de Feeback de entregable P0
## 5     5 Reunion para coordinar el tema final para el estudio estadistico
## 6     6     Postulacion de propuestas de temas por parte de los miembros
## 7     7        Seleccion de propuestas de tema por parte de los miembros
## 8     8   Reunion con Profesor Jose Renom para feeback del tema escogido
## 9     9            Reunion de coordinacion de parametros del experimento
## 10   10                  Aplicacion del experimentos al publico objetivo
## 11   11                                          Filtracion de data en R
## 12   12                                   Desarrollo de Informe en Rnote
## 13   13               Desarrollo de Presentacion Oral en R Presentations
## 14   14                       Entrega de Informe y Presentacion a Canvas
##         Elemento.del.Proyecto     Inicio        Fin  Color  X
## 1               Entregable P0 2022.03.21 2022.03.29    red NA
## 2               Entregable P0 2022.04.03 2022.04.06    red NA
## 3               Entregable P0 2022.04.06 2022.04.08    red NA
## 4               Entregable P0 2022.04.09 2022.04.21    red NA
## 5       Tema de Investigacion 2022.04.28 2022.04.30   blue NA
## 6       Tema de Investigacion 2022.05.02 2022.05.04   blue NA
## 7       Tema de Investigacion 2022.05.05 2022.05.05   blue NA
## 8  Parametros del experimento 2022.04.27 2022.04.29  green NA
## 9  Parametros del experimento 2022.05.01 2022.05.03  green NA
## 10 Desarrollo del Experimento 2022.05.02 2022.05.05 yellow NA
## 11              Entregable P1 2022.05.05 2022.05.06 purple NA
## 12              Entregable P1 2022.05.05 2022.05.06 purple NA
## 13              Entregable P1 2022.05.05 2022.05.07 purple NA
## 14              Entregable P1 2022.05.07 2022.05.07 purple NA
\end{verbatim}

\begin{Shaded}
\begin{Highlighting}[]
\NormalTok{acts }\OtherTok{\textless{}{-}} \FunctionTok{c}\NormalTok{(}\StringTok{"Primer contacto con equipo de trabajo"}\NormalTok{, }\StringTok{"Reunion para definir temas de posible estudio estadistico"}\NormalTok{,}\StringTok{"Redaccion de informe para el entregable P0"}\NormalTok{, }\StringTok{"Espera de Feeback de entregable P0"}\NormalTok{, }\StringTok{"Reunion para coordinar el tema final para el estudio estadistico"}\NormalTok{, }\StringTok{"Postulacion de propuestas de temas por parte de los miembros"}\NormalTok{, }\StringTok{"Seleccion de propuestas de tema por parte de los miembros"}\NormalTok{, }\StringTok{"Reunion con Profesor Jose Renom para feeback del tema escogido"}\NormalTok{, }\StringTok{"Reunion de coordinacion de parametros del experimento"}\NormalTok{, }\StringTok{"Aplicacion del experimentos al publico objetivo"}\NormalTok{, }\StringTok{"Filtracion de data en R"}\NormalTok{, }\StringTok{"Desarrollo de Informe en Rnote"}\NormalTok{, }\StringTok{"Desarrollo de Presentacion Oral en R Presentations"}\NormalTok{, }\StringTok{"Entrega de Informe y Presentacion a Canvas"}\NormalTok{)}
\NormalTok{els }\OtherTok{\textless{}{-}} \FunctionTok{c}\NormalTok{(}\StringTok{"Entregable P0"}\NormalTok{, }\StringTok{"Tema de Investigacion"}\NormalTok{, }\StringTok{"Parametros del experimento"}\NormalTok{, }\StringTok{"Desarrollo del Experimento"}\NormalTok{, }\StringTok{"Entregable P1"}\NormalTok{)}

\FunctionTok{length}\NormalTok{(acts)}
\end{Highlighting}
\end{Shaded}

\begin{verbatim}
## [1] 14
\end{verbatim}

\begin{Shaded}
\begin{Highlighting}[]
\NormalTok{g.gantt }\OtherTok{\textless{}{-}} \FunctionTok{gather}\NormalTok{(gantt, }\StringTok{"state"}\NormalTok{, }\StringTok{"date"}\NormalTok{, }\DecValTok{4}\SpecialCharTok{:}\DecValTok{5}\NormalTok{) }\SpecialCharTok{\%\textgreater{}\%} \FunctionTok{mutate}\NormalTok{(}\AttributeTok{date =} \FunctionTok{as.Date}\NormalTok{(date, }\StringTok{"\%Y.\%m.\%d"}\NormalTok{), }\AttributeTok{Actividad=}\FunctionTok{factor}\NormalTok{(Actividad, acts[}\FunctionTok{length}\NormalTok{(acts)}\SpecialCharTok{:}\DecValTok{1}\NormalTok{]), }\AttributeTok{Elemento.del.Proyecto =}\FunctionTok{factor}\NormalTok{(Elemento.del.Proyecto, els))}
\NormalTok{g.gantt}
\end{Highlighting}
\end{Shaded}

\begin{verbatim}
##    Item                                                        Actividad
## 1     1                            Primer contacto con equipo de trabajo
## 2     2        Reunion para definir temas de posible estudio estadistico
## 3     3                       Redaccion de informe para el entregable P0
## 4     4                               Espera de Feeback de entregable P0
## 5     5 Reunion para coordinar el tema final para el estudio estadistico
## 6     6     Postulacion de propuestas de temas por parte de los miembros
## 7     7        Seleccion de propuestas de tema por parte de los miembros
## 8     8   Reunion con Profesor Jose Renom para feeback del tema escogido
## 9     9            Reunion de coordinacion de parametros del experimento
## 10   10                  Aplicacion del experimentos al publico objetivo
## 11   11                                          Filtracion de data en R
## 12   12                                   Desarrollo de Informe en Rnote
## 13   13               Desarrollo de Presentacion Oral en R Presentations
## 14   14                       Entrega de Informe y Presentacion a Canvas
## 15    1                            Primer contacto con equipo de trabajo
## 16    2        Reunion para definir temas de posible estudio estadistico
## 17    3                       Redaccion de informe para el entregable P0
## 18    4                               Espera de Feeback de entregable P0
## 19    5 Reunion para coordinar el tema final para el estudio estadistico
## 20    6     Postulacion de propuestas de temas por parte de los miembros
## 21    7        Seleccion de propuestas de tema por parte de los miembros
## 22    8   Reunion con Profesor Jose Renom para feeback del tema escogido
## 23    9            Reunion de coordinacion de parametros del experimento
## 24   10                  Aplicacion del experimentos al publico objetivo
## 25   11                                          Filtracion de data en R
## 26   12                                   Desarrollo de Informe en Rnote
## 27   13               Desarrollo de Presentacion Oral en R Presentations
## 28   14                       Entrega de Informe y Presentacion a Canvas
##         Elemento.del.Proyecto  Color  X  state       date
## 1               Entregable P0    red NA Inicio 2022-03-21
## 2               Entregable P0    red NA Inicio 2022-04-03
## 3               Entregable P0    red NA Inicio 2022-04-06
## 4               Entregable P0    red NA Inicio 2022-04-09
## 5       Tema de Investigacion   blue NA Inicio 2022-04-28
## 6       Tema de Investigacion   blue NA Inicio 2022-05-02
## 7       Tema de Investigacion   blue NA Inicio 2022-05-05
## 8  Parametros del experimento  green NA Inicio 2022-04-27
## 9  Parametros del experimento  green NA Inicio 2022-05-01
## 10 Desarrollo del Experimento yellow NA Inicio 2022-05-02
## 11              Entregable P1 purple NA Inicio 2022-05-05
## 12              Entregable P1 purple NA Inicio 2022-05-05
## 13              Entregable P1 purple NA Inicio 2022-05-05
## 14              Entregable P1 purple NA Inicio 2022-05-07
## 15              Entregable P0    red NA    Fin 2022-03-29
## 16              Entregable P0    red NA    Fin 2022-04-06
## 17              Entregable P0    red NA    Fin 2022-04-08
## 18              Entregable P0    red NA    Fin 2022-04-21
## 19      Tema de Investigacion   blue NA    Fin 2022-04-30
## 20      Tema de Investigacion   blue NA    Fin 2022-05-04
## 21      Tema de Investigacion   blue NA    Fin 2022-05-05
## 22 Parametros del experimento  green NA    Fin 2022-04-29
## 23 Parametros del experimento  green NA    Fin 2022-05-03
## 24 Desarrollo del Experimento yellow NA    Fin 2022-05-05
## 25              Entregable P1 purple NA    Fin 2022-05-06
## 26              Entregable P1 purple NA    Fin 2022-05-06
## 27              Entregable P1 purple NA    Fin 2022-05-07
## 28              Entregable P1 purple NA    Fin 2022-05-07
\end{verbatim}

\begin{Shaded}
\begin{Highlighting}[]
\CommentTok{\#demosurv \textless{}{-} data.frame(state=rep(c("Inicio","Fin"), each=52), date=c(seq.Date(as.Date("2022.03.21"), as.Date("2022.05.06"), "3 week"), seq.Date(as.Date("2022.03.21"), as.Date("2022.05.06"), "3 week")), Actividad=rep(c(8.55,9.45), each=52),\textquotesingle{}Elemento del Proyecto\textquotesingle{}="", Item=rep(42:93, 2))}
\CommentTok{\#head(g.gantt$Actividad)}
\end{Highlighting}
\end{Shaded}

\begin{Shaded}
\begin{Highlighting}[]
\NormalTok{actcols }\OtherTok{\textless{}{-}} \FunctionTok{c}\NormalTok{(}\StringTok{"\#548235"}\NormalTok{, }\StringTok{"\#2E75B6"}\NormalTok{, }\StringTok{"\#BF9000"}\NormalTok{, }\StringTok{"\#7030A0"}\NormalTok{, }\StringTok{"\#cd6600"}\NormalTok{)}
\FunctionTok{ggplot}\NormalTok{(g.gantt, }\FunctionTok{aes}\NormalTok{(date, Actividad, }\AttributeTok{colour =}\NormalTok{ Elemento.del.Proyecto, }\AttributeTok{group=}\NormalTok{Item)) }\SpecialCharTok{+}
  \FunctionTok{geom\_line}\NormalTok{(}\AttributeTok{size =} \DecValTok{10}\NormalTok{) }\SpecialCharTok{+}
  \FunctionTok{scale\_color\_manual}\NormalTok{(}\AttributeTok{values=}\NormalTok{actcols, }\AttributeTok{name=}\StringTok{"Project component"}\NormalTok{) }\SpecialCharTok{+}
  \FunctionTok{labs}\NormalTok{(}\AttributeTok{x=}\StringTok{"Semana Ciclo 2022{-}1"}\NormalTok{, }\AttributeTok{y=}\ConstantTok{NULL}\NormalTok{, }\AttributeTok{title=}\StringTok{"Project timeline"}\NormalTok{) }\SpecialCharTok{+}
  \FunctionTok{scale\_x\_date}\NormalTok{(}\AttributeTok{breaks=}\FunctionTok{seq.Date}\NormalTok{(}\FunctionTok{as.Date}\NormalTok{(}\StringTok{"2022{-}03{-}21"}\NormalTok{), }\FunctionTok{as.Date}\NormalTok{(}\StringTok{"2022{-}5{-}06"}\NormalTok{), }\StringTok{"weeks"}\NormalTok{), }\AttributeTok{labels=}\FunctionTok{c}\NormalTok{(}\DecValTok{1}\NormalTok{,}\DecValTok{2}\NormalTok{,}\DecValTok{3}\NormalTok{,}\DecValTok{4}\NormalTok{,}\DecValTok{5}\NormalTok{,}\DecValTok{6}\NormalTok{,}\DecValTok{7}\NormalTok{)) }\SpecialCharTok{+}
  \FunctionTok{theme\_gray}\NormalTok{(}\AttributeTok{base\_size=}\DecValTok{14}\NormalTok{)}
\end{Highlighting}
\end{Shaded}

\includegraphics{proceso_files/figure-latex/unnamed-chunk-18-1.pdf}

\begin{center}\rule{0.5\linewidth}{0.5pt}\end{center}

\begin{Shaded}
\begin{Highlighting}[]
\NormalTok{DF }\OtherTok{\textless{}{-}} \FunctionTok{read\_csv}\NormalTok{(}\StringTok{"data.csv"}\NormalTok{)}
\end{Highlighting}
\end{Shaded}

\begin{verbatim}
## Rows: 85 Columns: 22
## -- Column specification --------------------------------------------------------
## Delimiter: ","
## chr (17): Submission Date, Ingrese su correo de UTEC!!, Nombre, Apellido, ¿C...
## dbl  (5): ¿Cual es tu edad?, ¿En que ciclo te encuentras?, ¿Cuál es tu nivel...
## 
## i Use `spec()` to retrieve the full column specification for this data.
## i Specify the column types or set `show_col_types = FALSE` to quiet this message.
\end{verbatim}

Filtrando todos los NA

\begin{Shaded}
\begin{Highlighting}[]
\NormalTok{DF }\OtherTok{\textless{}{-}}\NormalTok{ DF[}\FunctionTok{complete.cases}\NormalTok{(DF),]}
\end{Highlighting}
\end{Shaded}

Medir puntaje

\begin{Shaded}
\begin{Highlighting}[]
\NormalTok{score\_vector }\OtherTok{=} \FunctionTok{c}\NormalTok{()}

\ControlFlowTok{for}\NormalTok{ (ans }\ControlFlowTok{in}\NormalTok{ DF}\SpecialCharTok{$}\StringTok{\textasciigrave{}}\AttributeTok{Regula y proporciona la potencia para que se activen todos los componentes de la PC:}\StringTok{\textasciigrave{}}\NormalTok{)}
\NormalTok{\{}
  \ControlFlowTok{if}\NormalTok{ (}\SpecialCharTok{!}\FunctionTok{is.na}\NormalTok{(ans))}
\NormalTok{  \{}
    \ControlFlowTok{if}\NormalTok{ (ans }\SpecialCharTok{==} \StringTok{"Fuente de alimentación (PSU) o sistema de alimentación"}\NormalTok{)}
\NormalTok{    \{}
\NormalTok{      score\_vector }\OtherTok{\textless{}{-}} \FunctionTok{append}\NormalTok{(score\_vector, }\DecValTok{1}\NormalTok{)}
\NormalTok{    \}}
    \ControlFlowTok{else}
\NormalTok{    \{}
\NormalTok{      score\_vector }\OtherTok{\textless{}{-}} \FunctionTok{append}\NormalTok{(score\_vector, }\DecValTok{0}\NormalTok{)}
\NormalTok{    \}  }
\NormalTok{  \}}
  \ControlFlowTok{else}
\NormalTok{  \{}
\NormalTok{    score\_vector }\OtherTok{\textless{}{-}} \FunctionTok{append}\NormalTok{(score\_vector, }\DecValTok{0}\NormalTok{)}
\NormalTok{  \}}
  
\NormalTok{\}}

\NormalTok{iterator }\OtherTok{=} \DecValTok{1}
\ControlFlowTok{for}\NormalTok{ (ans }\ControlFlowTok{in}\NormalTok{ DF}\SpecialCharTok{$}\StringTok{\textasciigrave{}}\AttributeTok{Cuenta con sus propios procesadores y memoria interna que procesa los datos de la PC y los representa en forma de texto o gráficas complejas como la de los videojuegos. Nos referimos a:}\StringTok{\textasciigrave{}}\NormalTok{)}
\NormalTok{\{}
  \ControlFlowTok{if}\NormalTok{ (}\SpecialCharTok{!}\FunctionTok{is.na}\NormalTok{(ans))}
\NormalTok{  \{}
    \ControlFlowTok{if}\NormalTok{ (ans }\SpecialCharTok{==} \StringTok{"Tarjeta gráfica (GPU)"}\NormalTok{)}
\NormalTok{    \{}
\NormalTok{      score\_vector[iterator] }\OtherTok{=}\NormalTok{ score\_vector[iterator] }\SpecialCharTok{+} \DecValTok{1}
\NormalTok{    \}  }
\NormalTok{  \}}
  
\NormalTok{  iterator }\OtherTok{=}\NormalTok{ iterator }\SpecialCharTok{+} \DecValTok{1}
\NormalTok{\}}


\NormalTok{iterator }\OtherTok{=} \DecValTok{1}
\ControlFlowTok{for}\NormalTok{ (ans }\ControlFlowTok{in}\NormalTok{ DF}\SpecialCharTok{$}\StringTok{\textasciigrave{}}\AttributeTok{Responsable realiza cálculos basados en la interpretación de la información de ciertos componentes y les pasa el resultado a otros. Nos referimos a:}\StringTok{\textasciigrave{}}\NormalTok{)}
\NormalTok{\{}
  \ControlFlowTok{if}\NormalTok{ (}\SpecialCharTok{!}\FunctionTok{is.na}\NormalTok{(ans))}
\NormalTok{  \{}
    \ControlFlowTok{if}\NormalTok{ (ans }\SpecialCharTok{==} \StringTok{"Unidad central de procesamiento o Procesador (CPU)"}\NormalTok{)}
\NormalTok{    \{}
\NormalTok{      score\_vector[iterator] }\OtherTok{\textless{}{-}}\NormalTok{ score\_vector[iterator] }\SpecialCharTok{+} \DecValTok{1}
\NormalTok{    \}  }
\NormalTok{  \}}
  
\NormalTok{  iterator }\OtherTok{=}\NormalTok{ iterator }\SpecialCharTok{+} \DecValTok{1}
\NormalTok{\}}


\NormalTok{iterator }\OtherTok{=} \DecValTok{1}
\ControlFlowTok{for}\NormalTok{ (ans }\ControlFlowTok{in}\NormalTok{ DF}\SpecialCharTok{$}\StringTok{\textasciigrave{}}\AttributeTok{Almacena todos los datos del sistema operativo, los programas y las fotos, la música y los vídeos del usuario. Es decir, es el dispositivo de almacenamiento del ordenador.}\StringTok{\textasciigrave{}}\NormalTok{)}
\NormalTok{\{}
  \ControlFlowTok{if}\NormalTok{ (}\SpecialCharTok{!}\FunctionTok{is.na}\NormalTok{(ans))}
\NormalTok{  \{}
    \ControlFlowTok{if}\NormalTok{ (ans }\SpecialCharTok{==} \StringTok{"Disco Duro"}\NormalTok{)}
\NormalTok{    \{}
\NormalTok{      score\_vector[iterator] }\OtherTok{\textless{}{-}}\NormalTok{ score\_vector[iterator] }\SpecialCharTok{+} \DecValTok{1}
\NormalTok{    \}  }
\NormalTok{  \}}
  
\NormalTok{  iterator }\OtherTok{=}\NormalTok{ iterator }\SpecialCharTok{+} \DecValTok{1}
\NormalTok{\}}

\NormalTok{iterator }\OtherTok{=} \DecValTok{1}
\ControlFlowTok{for}\NormalTok{ (ans }\ControlFlowTok{in}\NormalTok{ DF}\SpecialCharTok{$}\StringTok{\textasciigrave{}}\AttributeTok{¿Dónde se conecta o instala la memoria (RAM)?}\StringTok{\textasciigrave{}}\NormalTok{)}
\NormalTok{\{}
  \ControlFlowTok{if}\NormalTok{ (}\SpecialCharTok{!}\FunctionTok{is.na}\NormalTok{(ans))}
\NormalTok{  \{}
    \ControlFlowTok{if}\NormalTok{ (ans }\SpecialCharTok{==} \StringTok{"Placa base (Motherboard)"}\NormalTok{)}
\NormalTok{    \{}
\NormalTok{      score\_vector[iterator] }\OtherTok{\textless{}{-}}\NormalTok{ score\_vector[iterator] }\SpecialCharTok{+} \DecValTok{1}
\NormalTok{    \}  }
\NormalTok{  \}}
  
\NormalTok{  iterator }\OtherTok{=}\NormalTok{ iterator }\SpecialCharTok{+} \DecValTok{1}
\NormalTok{\}}

\NormalTok{iterator }\OtherTok{=} \DecValTok{1}
\ControlFlowTok{for}\NormalTok{ (ans }\ControlFlowTok{in}\NormalTok{ DF}\SpecialCharTok{$}\StringTok{\textasciigrave{}}\AttributeTok{¿Dónde se coloca la pasta térmica?}\StringTok{\textasciigrave{}}\NormalTok{)}
\NormalTok{\{}
  \ControlFlowTok{if}\NormalTok{ (}\SpecialCharTok{!}\FunctionTok{is.na}\NormalTok{(ans))}
\NormalTok{  \{}
    \ControlFlowTok{if}\NormalTok{ (ans }\SpecialCharTok{==} \StringTok{"Unidad central de procesamiento o Procesador (CPU)"}\NormalTok{)}
\NormalTok{    \{}
\NormalTok{      score\_vector[iterator] }\OtherTok{\textless{}{-}}\NormalTok{ score\_vector[iterator] }\SpecialCharTok{+} \DecValTok{1}
\NormalTok{    \}  }
\NormalTok{  \}}
  
\NormalTok{  iterator }\OtherTok{=}\NormalTok{ iterator }\SpecialCharTok{+} \DecValTok{1}
\NormalTok{\}}

\NormalTok{iterator }\OtherTok{=} \DecValTok{1}
\ControlFlowTok{for}\NormalTok{ (ans }\ControlFlowTok{in}\NormalTok{ DF}\SpecialCharTok{$}\StringTok{\textasciigrave{}}\AttributeTok{¿A que parte se conecta el panel frontal de la caja (botón de encendido/apagado, USB, conector de audio, etc)?}\StringTok{\textasciigrave{}}\NormalTok{)}
\NormalTok{\{}
  \ControlFlowTok{if}\NormalTok{ (}\SpecialCharTok{!}\FunctionTok{is.na}\NormalTok{(ans))}
\NormalTok{  \{}
    \ControlFlowTok{if}\NormalTok{ (ans }\SpecialCharTok{==} \StringTok{"Placa base (Motherboard)"}\NormalTok{)}
\NormalTok{    \{}
\NormalTok{      score\_vector[iterator] }\OtherTok{\textless{}{-}}\NormalTok{ score\_vector[iterator] }\SpecialCharTok{+} \DecValTok{1}
\NormalTok{    \}  }
\NormalTok{  \}}
  
\NormalTok{  iterator }\OtherTok{=}\NormalTok{ iterator }\SpecialCharTok{+} \DecValTok{1}
\NormalTok{\}}



\NormalTok{DF\_scored }\OtherTok{\textless{}{-}} \FunctionTok{cbind}\NormalTok{(DF, score\_vector)}
\end{Highlighting}
\end{Shaded}

Ahora veamos la data.

\begin{Shaded}
\begin{Highlighting}[]
\NormalTok{class\_learning }\OtherTok{\textless{}{-}}\NormalTok{ DF\_scored[DF\_scored}\SpecialCharTok{$}\StringTok{\textasciigrave{}}\AttributeTok{Qué experiencia realizaste:}\StringTok{\textasciigrave{}} \SpecialCharTok{==} \StringTok{\textquotesingle{}Leer Documento PDF\textquotesingle{}}\NormalTok{,]}



\NormalTok{game\_learning }\OtherTok{\textless{}{-}}\NormalTok{ DF\_scored[DF\_scored}\SpecialCharTok{$}\StringTok{\textasciigrave{}}\AttributeTok{Qué experiencia realizaste:}\StringTok{\textasciigrave{}} \SpecialCharTok{==} \StringTok{\textquotesingle{}Jugar PC Simulator\textquotesingle{}}\NormalTok{,]}

\NormalTok{game\_learning\_cortado }\OtherTok{\textless{}{-}} \FunctionTok{head}\NormalTok{(game\_learning, }\FunctionTok{nrow}\NormalTok{(class\_learning) }\SpecialCharTok{{-}} \FunctionTok{nrow}\NormalTok{(game\_learning))}

\FunctionTok{nrow}\NormalTok{(game\_learning\_cortado)}
\end{Highlighting}
\end{Shaded}

\begin{verbatim}
## [1] 24
\end{verbatim}

\begin{Shaded}
\begin{Highlighting}[]
\NormalTok{data }\OtherTok{\textless{}{-}} \FunctionTok{data.frame}\NormalTok{(class\_learning}\SpecialCharTok{$}\NormalTok{score\_vector, game\_learning\_cortado}\SpecialCharTok{$}\NormalTok{score\_vector)}
\FunctionTok{boxplot}\NormalTok{(data,}\AttributeTok{names =} \FunctionTok{c}\NormalTok{(}\StringTok{"PDF"}\NormalTok{, }\StringTok{"Juego"}\NormalTok{), }\AttributeTok{col=}\FunctionTok{c}\NormalTok{(}\StringTok{"Red"}\NormalTok{, }\StringTok{"Blue"}\NormalTok{))}
\end{Highlighting}
\end{Shaded}

\includegraphics{proceso_files/figure-latex/unnamed-chunk-22-1.pdf}

Como podemos ver ahora hay una ligera inclinación a aquellos que
tuvieron la experiencia tradicional de leer el pdf en la clase. Esto se
puede deber a un sin fin de cosas, desde que aquellos seleccionados para
jugar el ``PC Building Simulator'' y llenaron el form eran personas con
menos tiempo hasta que decidieron llenarlo simplemente de forma no
honesta. Cabe decir de que la selección de grupos no fue de forma
aleatoria si no por conveniencia del entrevistador.

Por ultimo veamos las opiniones de los encuestados sobre su metodo
favorito

\begin{Shaded}
\begin{Highlighting}[]
\NormalTok{class\_learning }\OtherTok{\textless{}{-}}\NormalTok{ DF\_scored[DF\_scored}\SpecialCharTok{$}\StringTok{\textasciigrave{}}\AttributeTok{Qué experiencia realizaste:}\StringTok{\textasciigrave{}} \SpecialCharTok{==} \StringTok{\textquotesingle{}Leer Documento PDF\textquotesingle{}}\NormalTok{,]}



\NormalTok{game\_learning }\OtherTok{\textless{}{-}}\NormalTok{ DF\_scored[DF\_scored}\SpecialCharTok{$}\StringTok{\textasciigrave{}}\AttributeTok{Qué experiencia realizaste:}\StringTok{\textasciigrave{}} \SpecialCharTok{==} \StringTok{\textquotesingle{}Jugar PC Simulator\textquotesingle{}}\NormalTok{,]}

\NormalTok{game\_learning\_cortado }\OtherTok{\textless{}{-}} \FunctionTok{head}\NormalTok{(game\_learning, }\FunctionTok{nrow}\NormalTok{(class\_learning) }\SpecialCharTok{{-}} \FunctionTok{nrow}\NormalTok{(game\_learning))}


\NormalTok{data }\OtherTok{\textless{}{-}} \FunctionTok{data.frame}\NormalTok{(class\_learning}\SpecialCharTok{$}\StringTok{\textasciigrave{}}\AttributeTok{Nivel de satisfacción al realizar el juego o leer el documento pdf:}\StringTok{\textasciigrave{}}\NormalTok{, game\_learning\_cortado}\SpecialCharTok{$}\StringTok{\textasciigrave{}}\AttributeTok{Nivel de satisfacción al realizar el juego o leer el documento pdf:}\StringTok{\textasciigrave{}}\NormalTok{)}
\FunctionTok{boxplot}\NormalTok{(data,}\AttributeTok{names =} \FunctionTok{c}\NormalTok{(}\StringTok{"PDF"}\NormalTok{, }\StringTok{"Juego"}\NormalTok{))}
\end{Highlighting}
\end{Shaded}

\includegraphics{proceso_files/figure-latex/unnamed-chunk-23-1.pdf}

\begin{Shaded}
\begin{Highlighting}[]
\FunctionTok{mean}\NormalTok{(class\_learning}\SpecialCharTok{$}\StringTok{\textasciigrave{}}\AttributeTok{Nivel de satisfacción al realizar el juego o leer el documento pdf:}\StringTok{\textasciigrave{}}\NormalTok{)}
\end{Highlighting}
\end{Shaded}

\begin{verbatim}
## [1] 4.041667
\end{verbatim}

\begin{Shaded}
\begin{Highlighting}[]
\FunctionTok{mean}\NormalTok{(game\_learning\_cortado}\SpecialCharTok{$}\StringTok{\textasciigrave{}}\AttributeTok{Nivel de satisfacción al realizar el juego o leer el documento pdf:}\StringTok{\textasciigrave{}}\NormalTok{)}
\end{Highlighting}
\end{Shaded}

\begin{verbatim}
## [1] 4.708333
\end{verbatim}

Podemos observar de que hay una diferencia inmensa en la opinión de los
encuestados entre la clase tradicional y los que aprenden utilizando un
videojuego. habiendo una gran preferencia a los videojuegos.

En conclusión por la escala del proyecto no podemos realmente asegurar
alguna fuerte inclinación ni para la gamificación ni para la educación
tradicional. Aunque nuestro estudio sugiere de que la educación
tradicional parece ser mejor la diferencia es mínima y no es convulsiva.
Tambien el estudio sugiere de que hay una buena preferencia por la
Gamificación por parte de los estudiantes. Se necesitan mas estudios de
una mayor amplitud para tener una conclusión definitiva.

\begin{Shaded}
\begin{Highlighting}[]
\NormalTok{DF\_scored}\SpecialCharTok{$}\StringTok{\textasciigrave{}}\AttributeTok{Qué experiencia realizaste:}\StringTok{\textasciigrave{}} \OtherTok{\textless{}{-}} \FunctionTok{ifelse}\NormalTok{(DF\_scored}\SpecialCharTok{$}\StringTok{\textasciigrave{}}\AttributeTok{Qué experiencia realizaste:}\StringTok{\textasciigrave{}} \SpecialCharTok{==} \StringTok{"Leer Documento PDF"}\NormalTok{, }\StringTok{"Leer"}\NormalTok{, }\StringTok{"Jugar"}\NormalTok{)}

\NormalTok{.MinaSeg }\OtherTok{\textless{}{-}} \ControlFlowTok{function}\NormalTok{(x)\{}
\NormalTok{strs }\OtherTok{\textless{}{-}} \FunctionTok{strsplit}\NormalTok{(x, }\StringTok{":"}\NormalTok{)[[}\DecValTok{1}\NormalTok{]]}
\NormalTok{s }\OtherTok{\textless{}{-}} \FunctionTok{sum}\NormalTok{(}\FunctionTok{as.numeric}\NormalTok{(strs)}\SpecialCharTok{*}\FunctionTok{c}\NormalTok{(}\DecValTok{60}\NormalTok{, }\DecValTok{1}\NormalTok{))}
\FunctionTok{return}\NormalTok{(s)}
\NormalTok{\}}

\NormalTok{MinaSeg }\OtherTok{\textless{}{-}} \FunctionTok{Vectorize}\NormalTok{(.MinaSeg)}

\FunctionTok{boxplot}\NormalTok{(}\FunctionTok{MinaSeg}\NormalTok{(DF\_scored}\SpecialCharTok{$}\StringTok{\textasciigrave{}}\AttributeTok{Duración de la experiencia}\StringTok{\textasciigrave{}}\NormalTok{) }\SpecialCharTok{\textasciitilde{}}\NormalTok{ DF\_scored}\SpecialCharTok{$}\StringTok{\textasciigrave{}}\AttributeTok{Qué experiencia realizaste:}\StringTok{\textasciigrave{}}\SpecialCharTok{:}\NormalTok{DF\_scored}\SpecialCharTok{$}\NormalTok{score\_vector, }\AttributeTok{varwidth =} \ConstantTok{TRUE}\NormalTok{, }\AttributeTok{las =} \DecValTok{3}\NormalTok{, }\AttributeTok{cex.axis =} \FloatTok{0.6}\NormalTok{, }\AttributeTok{xlab =} \StringTok{"Actividad : Nota"}\NormalTok{, }\AttributeTok{ylab =} \StringTok{"tiempo (s)"}\NormalTok{)}
\end{Highlighting}
\end{Shaded}

\begin{verbatim}
## Warning in (function (x) : NAs introducidos por coerción

## Warning in (function (x) : NAs introducidos por coerción

## Warning in (function (x) : NAs introducidos por coerción

## Warning in (function (x) : NAs introducidos por coerción

## Warning in (function (x) : NAs introducidos por coerción

## Warning in (function (x) : NAs introducidos por coerción
\end{verbatim}

\includegraphics{proceso_files/figure-latex/unnamed-chunk-24-1.pdf}

\begin{Shaded}
\begin{Highlighting}[]
\FunctionTok{boxplot}\NormalTok{(DF\_scored}\SpecialCharTok{$}\NormalTok{score\_vector }\SpecialCharTok{\textasciitilde{}}\NormalTok{ DF\_scored}\SpecialCharTok{$}\StringTok{\textasciigrave{}}\AttributeTok{¿Cuál es tu nivel de conocimiento (previo a la experiencia) sobre componentes de computadora?}\StringTok{\textasciigrave{}}\SpecialCharTok{:}\NormalTok{DF\_scored}\SpecialCharTok{$}\StringTok{\textasciigrave{}}\AttributeTok{Qué experiencia realizaste:}\StringTok{\textasciigrave{}}\NormalTok{, }\AttributeTok{las =} \DecValTok{3}\NormalTok{, }\AttributeTok{cex.axis =} \FloatTok{0.6}\NormalTok{, }\AttributeTok{ylab =} \StringTok{"Nota"}\NormalTok{, }\AttributeTok{xlab =} \StringTok{"Conocimiento : Actividad"}\NormalTok{, }\AttributeTok{varwidth =} \ConstantTok{TRUE}\NormalTok{)}
\end{Highlighting}
\end{Shaded}

\includegraphics{proceso_files/figure-latex/unnamed-chunk-25-1.pdf}

Dado que el quiz tiene 8 preguntas, tomaremos una nota aprovatoria como
obtener de forma correcta al menos 5 de estas por las siguientes
razones: 1. Por que el minimo de nota aprovatoria regular (tanto de
universidad como de colegio) en el Perú es el 11/20 dado que 10/20 1/2
no seria considerado una nota aprovatoria en la universidad no
consideramos a los que tengan 4 o menos preguntas correctas como si
hubieran pasado el quiz. 2. Todas las preguntas tienen el mismo peso 3.
Competitividad un numero razonable de alumnos como podemos ver en el
promedio

\begin{Shaded}
\begin{Highlighting}[]
\NormalTok{passed }\OtherTok{\textless{}{-}} \FunctionTok{c}\NormalTok{(DF\_scored}\SpecialCharTok{$}\NormalTok{score\_vector }\SpecialCharTok{\textgreater{}} \DecValTok{4}\NormalTok{)}
\NormalTok{avg }\OtherTok{\textless{}{-}} \FunctionTok{sum}\NormalTok{(passed }\SpecialCharTok{==} \ConstantTok{TRUE}\NormalTok{)}\SpecialCharTok{/}\FunctionTok{length}\NormalTok{(passed)}
\NormalTok{avg}
\end{Highlighting}
\end{Shaded}

\begin{verbatim}
## [1] 0.4126984
\end{verbatim}

pasaron el quiz por lo que consideramos una buena forma de medir el
exito de cualquiera de los dos modos de estudio.

Ahora podemos ver un grafico del modelo binomial del numero de
estudiantes que se espera que pasaron el curso

\begin{Shaded}
\begin{Highlighting}[]
\FunctionTok{plot}\NormalTok{(}\FunctionTok{dbinom}\NormalTok{(}\DecValTok{1}\SpecialCharTok{:}\DecValTok{30}\NormalTok{,}\DecValTok{30}\NormalTok{,avg))}
\end{Highlighting}
\end{Shaded}

\includegraphics{proceso_files/figure-latex/unnamed-chunk-27-1.pdf}

Con una esperanza de:

\begin{Shaded}
\begin{Highlighting}[]
\NormalTok{Esperanza }\OtherTok{\textless{}{-}} \FunctionTok{length}\NormalTok{(passed)}\SpecialCharTok{*}\NormalTok{avg}
\NormalTok{Esperanza}
\end{Highlighting}
\end{Shaded}

\begin{verbatim}
## [1] 26
\end{verbatim}

\end{document}
